\chapter{Additional Programs \label{programs}}

Aspen comes with two helper programs.

\section{aspen.mod_wsgi \label{aspen.mod-wsgi}}

You can serve an Aspen website using the mod_wsgi extension module for Apache
via the \program{aspen.mod_wsgi} script. Since the same Aspen website can be
served by the main \program{aspen} program as by mod_wsgi, this gives you a
compelling development/deployment scenario.

To use mod_wsgi with Aspen, add the following two directives to your
\file{httpd.conf}:

\begin{verbatim}
WSGIScriptAlias / "\path\to\aspen.mod_wsgi"
SetEnv aspen.root "\path\to\website-root"
\end{verbatim}


\begin{seealso}

\seelink{http://code.google.com/p/modwsgi/}
{\code{mod_wsgi}}{The mod_wsgi Apache extension module}

\end{seealso}


\section{aspen.monitord \label{aspen.monitord}}

Aspen includes a daemon called \program{aspen.monitord} that launches the main
\program{aspen} daemon, and restarts it if it ever goes down. It takes one
command-line argument, the root filesystem path of the website to be monitored.
It uses the \file{__/var/aspen.pid} file to keep track of the main
\program{aspen} process, and it stores its own pid in
\file{__/var/aspen.monitord.pid}. \program{aspen.monitord} logs via the
\var{USER} syslog facility with an ident of \var{aspen.monitord}.
