\chapter{User Interface (UI) \label{interface}}

Users interface with Aspen through three mechanisms: the command line, a
configuration file, and the environment. Where a program parameter is set in
more than one of these contexts, they take precedence in the order given here.
For example, a \var{mode} option on the command line will override any
\var{mode} setting in the config file or in the environment.


\section{Command Line \label{command-line}}

Usage:

\begin{verbatim}
aspen [options] [command]
\end{verbatim}

Aspen takes one optional positional argument, \var{command}, which must be one
of: \code{start}, \code{status}, \code{stop}, \code{restart}, or \code{runfg}.
The default is \code{runfg}, which causes Aspen to run in the foreground,
sending all messages to stdout.

\code{start}, \code{status}, \code{stop}, and \code{restart} control Aspen as a
daemon, via a pidfile. If the website root has a directory named \file{__}
(that's two underscores; the \dfn{magic directory}), then the pidfile is at
\file{__/var/aspen.pid}. Otherwise, the pidfile is created in \file{/tmp}. When
run as a daemon, stdout and stderr are redirected to \file{__/var/aspen.log} if
\file{__} exists, and to \file{/dev/null} otherwise. The \file{__/var} directory
will be created if it does not exist. The permission mode of the pidfile is set
to \code{0600}; likewise with the logfile, unless it is \file{/dev/null}.

The Aspen distribution includes a script in \file{etc/aspen_bash_completion}
that can be used to configure the bash shell to autocomplete from among Aspen's
arguments. See the source for more information.

Aspen's command-line options are as follows:
% a subset of the options available in the config file:

\begin{tableiii}{l|l|l}{var}{Option}{Description}{Default}

\lineiii{\programopt{-a}/\longprogramopt{-address}=\var{address}}
    {The address to which Aspen should bind. If \var{address} begins with a dot
    or a forward slash, then it is interpreted as an AF_UNIX socket. If it
    contains more than one colon, it is seen as an AF_INET6 address. Otherwise,
    it is interpreted as AF_INET. If \var{address} begins with a colon, then the
    AF_INET loopback address is assumed.} {\code{0.0.0.0:8080}}

% \lineiii{\programopt{-l}/\longprogramopt{-log_filter}=\var{log_filter}}
%     {A subsystem filter to apply to the error log, per the logging module.}
%     {\code{}}

\lineiii{\programopt{-m}/\longprogramopt{-mode}=\var{mode}}
    {One of \code{debugging}, \code{development}, \code{staging}, or
    \code{production}. In debugging and development modes, Aspen will restart
    itself any time configuration files or module source files change on the
    filesystem.}
    {\code{development}}

\lineiii{\programopt{-r}/\longprogramopt{-root}=\var{root}}
    {The directory containing the website for Aspen to serve.}
    {\code{.}}

% \lineiii{\programopt{-v}/\longprogramopt{-log_level}=\var{log_level}}
%     {The error log level. Valid options per the logging module
%     are (case-insensitive): \code{notset}, \code{debug}, \code{info},
%     \code{warning}, \code{error}, \code{critical}.}{\code{warning}}

\end{tableiii}

\begin{seealso}

\seelink{http://zetadev.com/software/lib537/1.0a1/doc/html/module-mode.html}
{mode}{Aspen uses the \module{mode} module to model the application life-cycle.
It is available to your applications at \module{aspen.mode}}

\end{seealso}


\section{Configuration File \label{config-file}}

This section describes the general Aspen configuration file at
\file{__/etc/aspen.conf}. Additional configuration files are described in the
"Extending Aspen" chapter. \file{aspen.conf} is in \file{.ini}-style format per
the \module{ConfigParser} module. Aspen responds to the following settings in
the \code{main} section. You may define additional settings and sections that
are meaningful to your application, which you may access using the
\class{aspen.conf} object described below in the "API" chapter.

% There are five sections recognized: \code{DEFAULT},
% \code{debugging}, \code{development}, \code{staging}, and \code{production}. Any
% of the below settings can be given in any section, except for \var{mode}, which
% can only occur in \code{DEFAULT}. However, only two sections will be used at any
% given time: \code{DEFAULT}, and the section corresponding to the current
% deployment mode (see The Environment for more on mode).

\begin{tableiii}{l|l|l}{var}{Option}{Description}{Default}

\lineiii{address}{The address to which Aspen should bind. If \var{address}
begins with a dot or a forward slash, then it is interpreted as an AF_UNIX
socket. If it contains more than one colon, it is seen as an AF_INET6 address.
Otherwise, it is interpreted as AF_INET. If \var{address} begins with a colon,
then the AF_INET loopback address is assumed.}{\code{0.0.0.0:8080}}

\lineiii{defaults}{A comma-separated list of names to look for when a directory
is requested. Any default resource is located immediately before dispatching to
a handler.}{\code{index.html, index.htm}}

% \lineiii{group}{A groupname or gid to which, if given, Aspen will attempt to
% switch after binding to the socket.}{}
%
% \lineiii{log_access}{Whether or not to maintain an access log. Valid options
% are (case-insensitive): \code{yes}, \code{no}, \code{none}, \code{true},
% \code{false}, \code{0}, \code{1}. The access log will be in Apache's Combined
% Log Format.}{\code{no}}
%
% \lineiii{log_format}{The format of error log messages, per the logging
% module.}{\code{\%(levelname)s:\%(name)s:\%(message)s}}
%
% \lineiii{log_level}{The error log level. Valid options per the logging module
% are (case-insensitive): \code{notset}, \code{debug}, \code{info},
% \code{warning}, \code{error}, \code{critical}.}{\code{warning}}
%
% \lineiii{log_filter}{A subsystem filter to apply to the error log, per the
% logging module.}{}

\lineiii{mode}{One of \code{debugging}, \code{development}, \code{staging}, or
\code{production}. In debugging and development modes, Aspen will restart itself
any time configuration files or module source files change on the
filesystem.}{\code{development}}
% (Naturally, this option only obtains in the DEFAULT section.)

\lineiii{server_name}{The value to use for the \envvar{SERVER_NAME} WSGI
environment setting. }{\code{socket.gethostname()}}

\lineiii{threads}{The number of threads to maintain in the request-servicing
thread pool.}{\code{10}}

% \lineiii{user}{A username or uid to which, if given, Aspen will attempt to
% switch after binding to the socket.}{}
%
\end{tableiii}

\begin{seealso}

\seelink{extending.html}
{Extending Aspen}{Aspen three additional configuration files are described
here.}

\seelink{http://zetadev.com/software/lib537/1.0a1/doc/html/module-mode.html}
{\code{mode}}{Aspen relies on this module to model the application life-cycle.
It is available to your applications at \module{aspen.mode}}

\end{seealso}


\section{The Environment \label{environment}}

Aspen incorporates the \module{mode} module, which uses the \envvar{PYTHONMODE}
environment variable to model the application life-cycle through four deployment
modes: \code{debugging}, \code{development}, \code{staging}, and
\code{production}. This module is available to your applications at
\module{aspen.mode}, and its API is documented in the "API" chapter, below.

Aspen itself adapts to the current \envvar{PYTHONMODE}. In debugging and
development modes, Aspen will restart itself any time the following
configuration files or any module source files change on the filesystem:

\begin{itemize}
\item{\file{apps.conf}}
\item{\file{aspen.conf}}
\item{\file{handlers.conf}}
\item{\file{middleware.conf}}
\end{itemize}


\begin{seealso}

\seelink{http://zetadev.com/software/lib537/1.0a1/doc/html/module-mode.html}
{\code{mode}}{Aspen relies on this module to model the application life-cycle.
It is available to your applications at \module{aspen.mode}}

\end{seealso}
